\documentclass{article}
\usepackage{ctex}
\usepackage{graphicx}
\usepackage{float}
\usepackage{fontspec}
%\setmainfont[Mapping=tex-text]{KaiTi}
\title{\heiti CS61B课程笔记}
\author{黄毅男}
\date{}
\begin{document}
	\maketitle
	\tableofcontents
	\section{Java编程基础}
	Java的所有code都必须定义在类中。类中定义的变量和函数分static和non-static。static变量/函数是抽象的,可以由类本身直接引用。non-static变量/函数是具体的,必须先定义一个对象,然后对象才能调用non-static变量/函数。简单来说,当想要用类来调用某个成员时,请用static定义该成员;当想用具体的对象来调用某个成员时,用non-static定义该成员。
	
	特别的,java中的main函数也必须以static的方式定义在类中。
\end{document}