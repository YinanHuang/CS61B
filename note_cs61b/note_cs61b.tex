\documentclass{article}
\usepackage{ctex}
\usepackage{graphicx}
\usepackage{float}
\usepackage{fontspec}
%\setmainfont[Mapping=tex-text]{KaiTi}
\title{\heiti CS61B课程笔记}
\author{黄毅男}
\date{}
\begin{document}
	\maketitle
	\tableofcontents
	\section{Java编程基础}
	\subsection{Class in Java}
	Java的所有code都必须定义在类中。类中定义的变量和函数分static和non-static。static变量/函数是抽象的,可以由类本身直接引用。non-static变量/函数是具体的,必须先定义一个对象,然后对象才能调用non-static变量/函数。简单来说,当想要用类来调用某个成员时,请用static定义该成员;当想用具体的对象来调用某个成员时,用non-static定义该成员。
	
	特别的,java中的main函数也必须以static的方式定义在类中。
	\subsection{Private 与 public/Static与non-static}
	Private的变量只能在class内部调用,而public的变量可以在class外部调用。
	
	static指的是类本身的变量,通过类来调用;non-static必须通过对象调用。当定义类中的内部类时,若内部类为static,则这个内部类只能通过大类来声明、调用,故没办法访问大类中的non-static变量;当内部类是non-static时,需要通过大类的对象声明、调用。内部的non-static类不能有static变量。举个例子:
	
	(1)Outerclass里的Innerclass为static的。这时用Outerclass.Innerclass表示这个类,Innerclass的对象声明为new Outerclass.Innerclass(); Innerclass内的static变量x可以用Outerclass.Innerclass.x表示;non-static变量y用in=new Outerclass.Innerclass(), in.y表示。
	
	(2)Outerclass里的Innerclass为non-static的。这时还是用Outerclass.Interclass表示这个类,Innerclass的实例声明为 out.new Innerclass,out为Outerclass的实例。Innerclass内不能有static变量x;Innerclass内的non-static变量y用out.in.y表示。
	
	简单来说,static变量的前一级是类;non-static变量的前一级是对象。
	\subsection{JUnit Testing}
	对于类中的每个方法,都可以单独写一个test函数。这个test函数用non-static定义,然后在函数前加上@org.junit.Test,可以直接运行test函数。同时不同的test的独立进行。
\end{document}